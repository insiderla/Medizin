% tumabakus.sty
% 
% Kommando \abakus{<len>} zur Erzeugung des Abakus (Symbol der
% Fakultaet fuer Informatik, Technische Universitaet
% Muenchen). Ergebnis ist eine Box (LaTeX-Picture-Environment).
%
% Das Argument <len> gibt die Seitenlaenge des (quadratischen) Symbols
% an. Es muss eine Laengenangabe sein.
%
% Bsp:
%   \abakus{2.4cm}
%   \abakus{100pt}

\newcommand{\abakus}[1]{%
{\unitlength #1
\unitlength 0.55555\unitlength
\begin{picture}(1.8,1.8)
  \thicklines
  \multiput(0,0)(0.01,0){1}{\line(0,1){1.8}}
  \multiput(0,0)(0,0.01){1}{\line(1,0){1.8}}
  \multiput(1.8,0)(-0.01,0){1}{\line(0,1){1.8}}
  \multiput(0,1.8)(0,-0.01){1}{\line(1,0){1.8}}
  \multiput(0,1.2)(0,0.01){1}{\line(1,0){1.8}}

  \put(0.36,0){\line(0,1){0.115}}
  \put(0.36,0.285){\line(0,1){0.03}}
  \put(0.72,0){\line(0,1){0.315}}
  \put(0.72,0.685){\line(0,1){0.03}}
  \put(1.44,0.485){\line(0,1){0.23}}
  \put(0.36,1.085){\line(0,1){0.43}}
  \put(0.36,1.685){\line(0,1){0.115}}

  \multiput(0.36,0.485)(0.36,0){3}{\line(0,1){0.03}}
  \multiput(0.36,0.685)(0.72,0){2}{\line(0,1){0.23}}
  \multiput(0.72,0.885)(0.72,0){2}{\line(0,1){0.03}}
  \multiput(1.08,0)(0.36,0){2}{\line(0,1){0.115}}
  \multiput(1.08,0.285)(0.36,0){2}{\line(0,1){0.03}}
  \multiput(0.72,1.085)(0.36,0){3}{\line(0,1){0.23}}
  \multiput(0.72,1.485)(0.36,0){3}{\line(0,1){0.315}}

  \multiput(0.36,1.0)(0.36,0){4}{\circle{0.17}}
  \multiput(0.36,0.2)(0,0.2){3}{\circle{0.17}}
  \multiput(0.72,0.4)(0,0.2){3}{\circle{0.17}}
  \multiput(1.08,0.2)(0,0.2){3}{\circle{0.17}}
  \multiput(1.44,0.2)(0,0.2){2}{\circle{0.17}}
  \put(1.44,0.8){\circle{0.17}}
  \multiput(0.72,1.4)(0.36,0){3}{\circle{0.17}}
  \put(0.36,1.6){\circle{0.17}}
\end{picture}}}

